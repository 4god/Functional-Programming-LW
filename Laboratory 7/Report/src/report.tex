% !TEX root = main.tex

\section{Итеративный вариант функции \texttt{memberp}, которая возвращает \texttt{T}, или \texttt{NIL} в зависимости от того, принадлежит ли первый аргумент второму, как элемент}

\biglisting{../../Problems/src/problem-1.lisp}



\section{Итеративный вариант функции \texttt{assoc}}

\biglisting{../../Problems/src/problem-2.lisp}



\section{Итеративный вариант функции \texttt{length}}

\biglisting{../../Problems/src/problem-3.lisp}



\section{Итеративный вариант функции \texttt{nth}}

\biglisting{../../Problems/src/problem-4.lisp}



\section{Итеративный вариант функции \texttt{reverse}}

\biglisting{../../Problems/src/problem-5.lisp}



\section{Итеративные варианты функций, вычисляющие объединение, разность и симметрическую разность двух множеств}

\biglisting{../../Problems/src/problem-6.lisp}



\section{Функция возвращающая наибольший элемент из списка чисел}

\subsection{с помощью \texttt{dolist}}

\biglisting{../../Problems/src/problem-7-1.lisp}


\subsection{с помощью \texttt{do}}

\biglisting{../../Problems/src/problem-7-2.lisp}



\section{Функция возвращающая первый нечисловой элемент из списка}

\subsection{с помощью \texttt{dolist}}

\biglisting{../../Problems/src/problem-8-1.lisp}


\subsection{с помощью \texttt{do}}

\biglisting{../../Problems/src/problem-8-2.lisp}


\subsection{с помощью рекурсии}

\biglisting{../../Problems/src/problem-8-3.lisp}



\section{Функция сортирующая список из чисел по возрастанию}

\subsection{итеративный способ}

\biglisting{../../Problems/src/problem-9-1.lisp}


\subsection{рекурсивный способ}

\biglisting{../../Problems/src/problem-9-2.lisp}