% !TEX root = main.tex

\section*{Полезные функции}

\biglisting{../../Problems/src/lib.lisp}



\section{Функция, которая по своему списку-аргументу \texttt{lst} определяет является ли он палиндромом (то есть равны ли \texttt{lst} и \texttt{'(reverse lst)})}

\biglisting{../../Problems/src/problem-1.lisp}



\section{Предикат \texttt{set-equal}, который возвращает \texttt{T}, если два его множества-аргу\-мента содержат одни и те же элементы, порядок которых не имеет значения}

\biglisting{../../Problems/src/problem-2.lisp}



\section{Функции, которые обрабатывают таблицу из точечных пар (страна . столица) и возвращают по стране --- столицу, а по столице --- страну}

\biglisting{../../Problems/src/problem-3.lisp}



\section{Функция, которая переставляет в списке-аргументе первый и последний элемент}


\subsection{с использованием \texttt{rplaca} и \texttt{rplacd}}

\biglisting{../../Problems/src/problem-4-1.lisp}


\subsection{с использованием \texttt{butlast}}

\biglisting{../../Problems/src/problem-4-2.lisp}


\subsection{с использованием \texttt{remove-if}}

\biglisting{../../Problems/src/problem-4-3.lisp}



\section{Функция, которая переставляет в списке-аргументе два указанных своими порядковыми номерами элемента в этом списке}

\biglisting{../../Problems/src/problem-5.lisp}



\section{Функции, которые производят круговую перестановку в списке-аргументе влево и вправо}

\subsection{циклический сдвиг влево}

\biglisting{../../Problems/src/problem-6-1.lisp}

\subsection{циклический сдвиг вправо}

\biglisting{../../Problems/src/problem-6-2.lisp}



\section{Функция, которая умножает на заданное число-аргумент все числа из заданного списка-аргумента}


\subsection{все элементы списка --- числа}

\biglisting{../../Problems/src/problem-7-1.lisp}

\subsection{элементы списка --- любые объекты}

\biglisting{../../Problems/src/problem-7-1.lisp}



\section{Функция, которая из списка-аргумента, содержащего только числа, выбирает только те, которые расположены между двумя указанными гра\-ни\-ца\-ми-аргументами и возвращает их в виде списка упорядоченного по возрастанию списка чисел}

\biglisting{../../Problems/src/problem-8.lisp}