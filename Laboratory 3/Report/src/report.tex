% !TEX root = main.tex

\section{Написать функцию}

\problem Функция принимает целое число и возвращает первое чётное число, не меньшее аргумента.

\biglisting{../../Problems/src/problem-1-1.lisp}


\problem Функция принимает число и возвращает число того же знака, но с модулем на единицу больше модуля аргумента.

\biglisting{../../Problems/src/problem-1-2.lisp}


\problem Функция принимает два числа и возвращает список из этих чисел отсортированный по возрастанию.

\biglisting{../../Problems/src/problem-1-3.lisp}


\problem Функция принимает три числа и возвращает \verb|T| если первое число расположено между вторым и третьим.

\biglisting{../../Problems/src/problem-1-4.lisp}



\section{Вычсилить результат выражений}

\problem \hfill
\begin{lstlisting}
(and 'fee 'fie 'foe) => FOE
\end{lstlisting}


\problem \hfill
\begin{lstlisting}
(or 'fee 'fie 'foe) => FEE
\end{lstlisting}