% !TEX root = main.tex

\section{Написать функцию}

\problem Функция принимает целое число и возвращает первое чётное число, не меньшее аргумента.

\biglisting{../../Problems/src/problem-1-1.lisp}


\problem Функция принимает число и возвращает число того же знака, но с модулем на единицу больше модуля аргумента.

\biglisting{../../Problems/src/problem-1-2.lisp}


\problem Функция принимает два числа и возвращает список из этих чисел отсортированный по возрастанию.

\biglisting{../../Problems/src/problem-1-3.lisp}


\problem Функция принимает три числа и возвращает \verb|T| если первое число расположено между вторым и третьим.\\

\noindent
Возможны следующие реализации данной функции:
\begin{itemize}
	\item с использованием функции \verb|SORT|;
	\biglisting{../../Problems/src/problem-1-4-1.lisp}
	
	\item с помощью специального оператора \verb|IF|;
	\biglisting{../../Problems/src/problem-1-4-2.lisp}
	
	\item c помощью макросов \verb|AND| и \verb|OR|;
	\biglisting{../../Problems/src/problem-1-4-3.lisp}
	
	\item c помощью макроса \verb|COND|.
	\biglisting{../../Problems/src/problem-1-4-4.lisp}
\end{itemize}




\section{Вычсилить результат выражений}

\problem \hfill
\begin{lstlisting}
(and 'fee 'fie 'foe) => FOE
\end{lstlisting}


\problem \hfill
\begin{lstlisting}
(or 'fee 'fie 'foe) => FEE
\end{lstlisting}


\problem \hfill
\begin{lstlisting}
(and (equal 'abc 'abc) 'yes) => YES
\end{lstlisting}


\problem \hfill
\begin{lstlisting}
(or nil 'fie 'foe) => FIE
\end{lstlisting}


\problem \hfill
\begin{lstlisting}
(and nil 'fie 'foe) => NIL
\end{lstlisting}


\problem \hfill
\begin{lstlisting}
(or (equal 'abc 'abc) 'yes) => T
\end{lstlisting}



\section{Написать предикат}

\problem
Предикат принимает два числа и возвращает \verb|T|, если первое число не меньше второго.

\biglisting{../../Problems/src/problem-3-1.lisp}



\section{Найти ошибку}

\problem Найти ошибочный предикат, Объяснить почему.

\biglisting{../../Problems/src/problem-4-1.lisp}

\noindent
Предикат \verb|pred2| является неправильным, так как данная реализация выдаст ошибку, если передать в качестве аргумента не число. Ошибка связана с тем, что в реализации \verb|pred2|, до проверки на то, что аргумент является числом, выполняется функция предназначенная для чисел.