% !TEX root = main.tex

\section{Представить списки в виде списочных ячеек}


\paragraph{Задание 1.1} $'(open\; close\; halph)$
\vfill

\begin{center}\begin{tikzpicture}
	\node [conscell] (cell1) {};
	\node [conscell, right= of cell1] (cell2) {};
	\node [conscell, right= of cell2] (cell3) {};
	
	\draw [pointer] (c2 cell1) -- (cell2);
	\draw [pointer] (c2 cell2) -- (cell3);
	\draw [pointer] (c2 cell3) -- +(1,  0) node [anchor=west] {Nil};
	\draw [pointer] (c1 cell1) -- +(0, -1) node [anchor=north] {open};
	\draw [pointer] (c1 cell2) -- +(0, -1) node [anchor=north] {close};
	\draw [pointer] (c1 cell3) -- +(0, -1) node [anchor=north] {halph};
	
	\foreach \i in {cell1,cell2,cell3}{\draw[fill=red] (c1 \i)circle(2pt) (c2 \i)circle(2pt);}
\end{tikzpicture}\end{center}


\paragraph{Задание 1.2} $'\bigl((open1)\;(close2)\;(halph3)\bigr)$
\vfill

\begin{center}\begin{tikzpicture}
	\node [conscell] (cell1) {};
	\node [conscell, right=2cm of cell1] (cell2) {};
	\node [conscell, right=2cm of cell2] (cell3) {};
	\node [conscell, below=    of cell1] (cell4) {};
	\node [conscell, below=    of cell2] (cell5) {};
	\node [conscell, below=    of cell3] (cell6) {};
	
	\draw [pointer] (c2 cell1) -- (cell2);
	\draw [pointer] (c2 cell2) -- (cell3);
	\draw [pointer] (c2 cell3) -- +(1, 0) node [anchor=west] {Nil};
	
	\draw [pointer] (c1 cell1) -- (cell4.135);
	\draw [pointer] (c1 cell2) -- (cell5.135);
	\draw [pointer] (c1 cell3) -- (cell6.135);
	
	\draw [pointer] (c1 cell4) -- +(0, -1) node [anchor=north] {open1};
	\draw [pointer] (c2 cell4) -- +(1,  0) node [anchor=west]  {Nil};
	
	\draw [pointer] (c1 cell5) -- +(0, -1) node [anchor=north] {close2};
	\draw [pointer] (c2 cell5) -- +(1,  0) node [anchor=west]  {Nil};
	
	\draw [pointer] (c1 cell6) -- +(0, -1) node [anchor=north] {halph3};
	\draw [pointer] (c2 cell6) -- +(1,  0) node [anchor=west]  {Nil};
	
	
	\foreach \i in {cell1,cell2,cell3,cell4,cell5,cell6}{\draw[fill=red] (c1 \i)circle(2pt) (c2 \i)circle(2pt);}
\end{tikzpicture}\end{center}
\vfill

\paragraph{Задание 1.3} $'\biggl((one)\; f\!or\; all\; \Bigl(and\; \bigl(me\; (f\!or\; you)\bigr)\Bigr)\biggr)$
\vfill

\begin{center}\begin{tikzpicture}
	\node [conscell] (cell1) {};
	\node [conscell, right=2cm of cell1] (cell2) {};
	\node [conscell, right=    of cell2] (cell3) {};
	\node [conscell, right=    of cell3] (cell4) {};
	\node [conscell, below=    of cell1] (cell5) {};
	\node [conscell, below=2cm of cell2] (cell6) {};
	\node [conscell, right=    of cell6] (cell7) {};
	\node [conscell, below=    of cell7] (cell8) {};
	\node [conscell, right=    of cell8] (cell9) {};
	\node [conscell, below=    of cell9] (cell10) {};
	\node [conscell, right=    of cell10] (cell11) {};
	
	
	\draw [pointer] (c2 cell1) -- (cell2);
	\draw [pointer] (c2 cell2) -- (cell3);
	\draw [pointer] (c2 cell3) -- (cell4);
	\draw [pointer] (c2 cell4) -- +(1, 0) node [anchor=west] {Nil};
	
	\draw [pointer] (c1 cell1) -- (cell5.135);
	\draw [pointer] (c1 cell2) -- +(0, -1) node [anchor=north] {for};
	\draw [pointer] (c1 cell3) -- +(0, -1) node [anchor=north] {all};
	\draw [-,thick] (c1 cell4) -- +(0, -2) node (t1) {};
	\draw [pointer] (t1.text)  -| (cell6.135);
	
	
	\draw [pointer] (c1 cell5) -- +(0, -1) node [anchor=north] {one};
	\draw [pointer] (c2 cell5) -- +(1, 0) node [anchor=west] {Nil};
	
	
	\draw [pointer] (c2 cell6) -- (cell7);
	\draw [pointer] (c2 cell7) -- +(1, 0) node [anchor=west] {Nil};
	
	\draw [pointer] (c1 cell6) -- +(0, -1) node [anchor=north] {and};
	\draw [pointer] (c1 cell7) -- (cell8.135);
	
	
	\draw [pointer] (c2 cell8) -- (cell9);
	\draw [pointer] (c2 cell9) -- +(1, 0) node [anchor=west] {Nil};
	
	\draw [pointer] (c1 cell8) -- +(0, -1) node [anchor=north] {me};
	\draw [pointer] (c1 cell9) -- (cell10.135);
	
	
	\draw [pointer] (c2 cell10) -- (cell11);
	\draw [pointer] (c2 cell11) -- +(1, 0) node [anchor=west] {Nil};
	
	\draw [pointer] (c1 cell10) -- +(0, -1) node [anchor=north] {for};
	\draw [pointer] (c1 cell11) -- +(0, -1) node [anchor=north] {you};
	
	
	\foreach \i in {cell1,cell2,cell3,cell4,cell5,cell6,cell7,cell8,cell9,cell10,cell11}{\draw[fill=red] (c1 \i)circle(2pt) (c2 \i)circle(2pt);}
\end{tikzpicture}\end{center}
\vfill


\newpage
\paragraph{Задание 1.4} $'\bigl((tool)\;(call)\bigr)$
\vfill

\begin{center}\begin{tikzpicture}
	\node [conscell] (cell1) {};
	\node [conscell, right=2cm of cell1] (cell2) {};
	\node [conscell, below=    of cell1] (cell3) {};
	\node [conscell, below=    of cell2] (cell4) {};
	
	\draw [pointer] (c2 cell1) -- (cell2);
	\draw [pointer] (c2 cell2) -- +(1, 0) node [anchor=west] {Nil};
	
	\draw [pointer] (c1 cell1) -- (cell3.135);
	\draw [pointer] (c1 cell2) -- (cell4.135);
	
	\draw [pointer] (c1 cell3) -- +(0, -1) node [anchor=north] {tool};
	\draw [pointer] (c2 cell3) -- +(1,  0) node [anchor=west]  {Nil};
	
	\draw [pointer] (c1 cell4) -- +(0, -1) node [anchor=north] {call};
	\draw [pointer] (c2 cell4) -- +(1,  0) node [anchor=west]  {Nil};
	
	
	\foreach \i in {cell1,cell2,cell3,cell4}{\draw[fill=red] (c1 \i)circle(2pt) (c2 \i)circle(2pt);}
\end{tikzpicture}\end{center}
\vfill
 

\paragraph{Задание 1.5} $'\Bigl((tool1)\;\bigl((call2)\bigr)\;\bigl((sell)\bigr)\Bigr)$
\vfill

\begin{center}\begin{tikzpicture}
	\node [conscell] (cell1) {};
	\node [conscell, right=2cm of cell1] (cell2) {};
	\node [conscell, right=2cm of cell2] (cell3) {};
	\node [conscell, below=    of cell1] (cell4) {};
	\node [conscell, below=    of cell2] (cell5) {};
	\node [conscell, below=    of cell3] (cell6) {};
	\node [conscell, below=    of cell5] (cell7) {};
	\node [conscell, below=    of cell6] (cell8) {};
	
	\draw [pointer] (c2 cell1) -- (cell2);
	\draw [pointer] (c2 cell2) -- (cell3);
	\draw [pointer] (c2 cell3) -- +(1, 0) node [anchor=west] {Nil};
	
	\draw [pointer] (c1 cell1) -- (cell4.135);
	\draw [pointer] (c1 cell2) -- (cell5.135);
	\draw [pointer] (c1 cell3) -- (cell6.135);
	
	\draw [pointer] (c1 cell4) -- +(0, -1) node [anchor=north] {tool1};
	\draw [pointer] (c2 cell4) -- +(1,  0) node [anchor=west]  {Nil};
	
	\draw [pointer] (c1 cell5) -- (cell7.135);
	\draw [pointer] (c2 cell5) -- +(1,  0) node [anchor=west]  {Nil};
	
	\draw [pointer] (c1 cell6) -- (cell8.135);
	\draw [pointer] (c2 cell6) -- +(1,  0) node [anchor=west]  {Nil};
	
	\draw [pointer] (c1 cell7) -- +(0, -1) node [anchor=north] {call2};
	\draw [pointer] (c2 cell7) -- +(1,  0) node [anchor=west]  {Nil};
	
	\draw [pointer] (c1 cell8) -- +(0, -1) node [anchor=north] {sell};
	\draw [pointer] (c2 cell8) -- +(1,  0) node [anchor=west]  {Nil};
	
	\foreach \i in {cell1,cell2,cell3,cell4,cell5,cell6,cell7,cell8}{\draw[fill=red] (c1 \i)circle(2pt) (c2 \i)circle(2pt);}
\end{tikzpicture}\end{center}
\vfill


\paragraph{Задание 1.6}$'\Bigl(\bigl((tool)\;(call)\bigr)\;\bigl((sell)\bigr)\Bigr)$
\vfill

\begin{center}\begin{tikzpicture}
	\node [conscell] (cell1) {};
	\node [conscell, right=6cm of cell1] (cell2) {};
	\node [conscell, below=    of cell1] (cell3) {};
	\node [conscell, right=2cm of cell3] (cell4) {};
	\node [conscell, below=    of cell2] (cell5) {};
	\node [conscell, below=    of cell3] (cell6) {};
	\node [conscell, below=    of cell4] (cell7) {};
	\node [conscell, below=    of cell5] (cell8) {};

	
	\draw [pointer] (c2 cell1) -- (cell2);
	\draw [pointer] (c2 cell2) -- +(1, 0) node [anchor=west] {Nil};

	\draw [pointer] (c1 cell1) -- (cell3.135);
	\draw [pointer] (c1 cell2) -- (cell5.135);

	
	\draw [pointer] (c2 cell3) -- (cell4);
	\draw [pointer] (c2 cell4) -- +(1, 0) node [anchor=west] {Nil};
	\draw [pointer] (c2 cell5) -- +(1, 0) node [anchor=west] {Nil};

	\draw [pointer] (c1 cell3) -- (cell6.135);
	\draw [pointer] (c1 cell4) -- (cell7.135);
	\draw [pointer] (c1 cell5) -- (cell8.135);

	
	\draw [pointer] (c2 cell6) -- +(1,  0) node [anchor=west]  {Nil};
	\draw [pointer] (c1 cell6) -- +(0, -1) node [anchor=north] {tool};
	
	
	\draw [pointer] (c2 cell7) -- +(1,  0) node [anchor=west]  {Nil};
	\draw [pointer] (c1 cell7) -- +(0, -1) node [anchor=north] {cool};
	
	
	\draw [pointer] (c2 cell8) -- +(1,  0) node [anchor=west]  {Nil};
	\draw [pointer] (c1 cell8) -- +(0, -1) node [anchor=north] {sell};


	\foreach \i in {cell1,cell2,cell3,cell4,cell5,cell6,cell7,cell8}{\draw[fill=red] (c1 \i)circle(2pt) (c2 \i)circle(2pt);}
\end{tikzpicture}\end{center}
\vfill



\newpage
\section{Используя только функции \texttt{CAR} и \texttt{CDR}, написать выражение}

\paragraph{Задание 2.1} Возвращающие второй элемент списка

\begin{lstlisting}
(cadr example-list)
\end{lstlisting}


\paragraph{Задание 2.2} Возвращающие третий элемент списка
\begin{lstlisting}
(caddr example-list)
\end{lstlisting}


\paragraph{Задание 2.3} Возвращающие четвёртый элемент списка
\begin{lstlisting}
(cadddr example-list)
\end{lstlisting}



\section{Напишите результат вычисления выражения}

\paragraph{Задание 3.1} \hfill

\begin{lstlisting}
(list 'Fred 'and 'Wilma) => (FRED AND WILMA)
\end{lstlisting}


\paragraph{Задание 3.2} \hfill

\begin{lstlisting}
(list 'Fred '(and Wilma)) => (FRED (AND WILMA))
\end{lstlisting}


\paragraph{Задание 3.3} \hfill

\begin{lstlisting}
(cons Nil Nil) => (NIL)
\end{lstlisting}


\paragraph{Задание 3.4} \hfill

\begin{lstlisting}
(cons T Nil) => (T)
\end{lstlisting}


\paragraph{Задание 3.5} \hfill

\begin{lstlisting}
(cons Nil T) => (NIL . T)
\end{lstlisting}