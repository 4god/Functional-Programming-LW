% !TEX root = main.tex

\section{Для указанных выражений составить диаграмму вычисления}

\vfill
\problem $\bigl(equal\; 3\; (abs -\!3)\bigr)$

\begin{itemize}
	\item[$\longrightarrow$]$\bigl(equal\; 3\; (abs -\!3)\bigr)$
	\begin{itemize}
		\item[\textbullet] $3 \to 3$;
		\item[$\longrightarrow$] $(abs -\!3)$:
		\begin{itemize}
			\item[\textbullet] $-3 \to -3$;
			\item[$\Longrightarrow$] применение $abs$ к $-3$;
			\item[$\Longrightarrow$] \underline{результат}: $3$;
		\end{itemize}
	\end{itemize}
	\item[$\Longrightarrow$] применение $equal$ к $3, 3$
	\item[$\Longrightarrow$] \underline{результат}: T.
\end{itemize}
\vfill


\problem $\bigl(equal\; (+\; 1\; 2)\; 3\bigr)$

\begin{itemize}
	\item[$\longrightarrow$] $\bigl(equal\; (+\; 1\; 2)\; 3\bigr)$:
	\begin{itemize}
		\item[$\longrightarrow$] $(+\; 1\; 2)$:
		\begin{itemize}
			\item[\textbullet] $1 \to 1$;
			\item[\textbullet] $2 \to 2$;
			\item[$\Longrightarrow$] применение <<$+$>> к $1, 2$;
			\item[$\Longrightarrow$] \underline{результат}: $3$;
		\end{itemize}
		\item[\textbullet] $3 \to 3$;
		\item[$\Longrightarrow$] применение $equal$ к $3, 3$;
		\item[$\Longrightarrow$] \underline{результат}: T.
	\end{itemize}
\end{itemize}
\vfill


\problem $\bigl(equal\; (*\; 4\; 7)\; 21\bigr)$

\begin{enumerate}
	\item $(*\; 4\; 7)$:
	\begin{enumerate}
		\item $4 \to 4$;
		\item $7 \to 7$;
		\item применение <<$*$>> к $4, 7$;
		\item \underline{результат}: $28$;
	\end{enumerate}
	\item $21 \to 21$;
	\item применение $equal$ к $28, 21$;
	\item \underline{результат}: NIL.
\end{enumerate}


\newpage
\problem $\bigl(equal\; (*\; 2\; 3)\; (+\; 7\; 2) \bigr)$

\begin{enumerate}
	\item $(*\; 2\; 3)$:
	\begin{enumerate}
		\item $2 \to 2$;
		\item $3 \to 3$;
		\item применение <<$*$>> к $2, 3$;
		\item \underline{результат}: $6$;
	\end{enumerate}
	\item $(+\; 7\; 2)$:
	\begin{enumerate}
		\item $7 \to 7$;
		\item $2 \to 2$;
		\item применение <<$+$>> к $7, 2$;
		\item \underline{результат}: $9$;
	\end{enumerate}
	\item применение $equal$ к $6, 9$;
	\item \underline{результат}: NIL.
\end{enumerate}


\problem $\bigl(equal\; (-\; 7\; 3)\; (*\; 3\; 2) \bigr)$

\begin{enumerate}
	\item $(-\; 7\; 3)$:
	\begin{enumerate}
		\item $7 \to 7$;
		\item $3 \to 3$;
		\item применение <<$-$>> к $7, 3$;
		\item \underline{результат}: $4$;
	\end{enumerate}
	\item $(*\; 3\; 2)$:
	\begin{enumerate}
		\item $3 \to 3$;
		\item $2 \to 2$;
		\item применение <<$*$>> к $3, 2$;
		\item \underline{результат}: $2$;
	\end{enumerate}
	\item применение $equal$ к $4, 6$;
	\item \underline{результат}: NIL.
\end{enumerate}


\problem $\Bigl(equal\; \bigl(abs\; (-\; 2\; 4)\bigr)\; 3\Bigr)$

\begin{enumerate}
	\item $\bigl(abs\; (-\; 2\; 4)\bigr)$:
	\begin{enumerate}
		\item $(-\; 2\; 4)$:
		\begin{enumerate}
			\item $2 \to 2$;
			\item $4 \to 4$;
			\item применение <<$-$>> к $2, 4$;
			\item \underline{результат}: $-2$;
		\end{enumerate}
		\item применение $abs$ к $-2$;
		\item \underline{результат}: $2$;
	\end{enumerate}
	\item $3 \to 3$;
	\item применение $equal$ к $2, 3$;
	\item \underline{результат}: NIL.
\end{enumerate}
\vfill



\section{Написать функцию и составить для неё диаграмму вычисления}

\problem Функция вычисляет гипотенузу прямоугольного треугольника по заданным катетам.
\vfill

\biglisting{../../Problems/src/problem-2-1.lisp}
\vfill

\[
	\Bigl(sqrt\; \bigl(+\; (*\; arg1\; arg1)\; (*\; arg2\; arg2)\bigr)\!\Bigr)
\]
\vfill

\begin{enumerate}
	\item $\bigl(+\; (*\; arg1\; arg1)\; (*\; arg2\; arg2)\bigr)$:
	\begin{enumerate}
		\item $(*\; arg1\; arg1)$:
		\begin{enumerate}
			\item $arg1 \to arg1$;
			\item $arg1 \to arg1$;
			\item применение <<$*$>> к $arg1, arg1$;
			\item \underline{результат}: $(arg1)^2$.
		\end{enumerate}
		\item $(*\; arg2\; arg2)$:
		\begin{enumerate}
			\item $arg2 \to arg2$;
			\item $arg2 \to arg2$;
			\item применение <<$*$>> к $arg2, arg2$;
			\item \underline{результат}: $(arg2)^2$.
		\end{enumerate}
		\item применение <<$+$>> к $(arg1)^2, (arg2)^2$;
		\item \underline{результат}: $(arg1)^2 + (arg2)^2$.
	\end{enumerate}
	\item применение $sqrt$ к $(arg1)^2 + (arg2)^2$;
	\item \underline{результат}: $\sqrt{(arg1)^2 + (arg2)^2}$.
\end{enumerate}


\newpage
\problem Функция вычисляет объём прямоугольного параллелепипеда по 3-м его сторонам.

\biglisting{../../Problems/src/problem-2-2.lisp}

\begin{itemize}
	\item[$\longrightarrow$] $(volume\!-\!rect\!-\!parallepiped\; q\; w\; e)$;
	\item $q \to q$;
	\item $w \to w$;
	\item $e \to e$;
	\item[$\Longrightarrow$] применение $volume\!-\!rect\!-\!parallepiped$ к $q, w, e$:
	\begin{itemize}
		\item[\textbullet] $arg1 \to q$;
		\item[\textbullet] $arg2 \to w$;
		\item[\textbullet] $arg3 \to e$;
		\item[$\longrightarrow$] $(*\; arg1\; arg2\; arg3)$:
		\begin{itemize}
			\item[\textbullet] $arg1 \to q$;
			\item[\textbullet] $arg2 \to w$;
			\item[\textbullet] $arg3 \to e$;
			\item[$\Longrightarrow$] применение <<$*$>> к $arg1, arg2, arg3$;
			\item[$\Longrightarrow$] \underline{результат}: $q \cdot w \cdot e$.
		\end{itemize}
	\end{itemize}
	\item[$\Longrightarrow$] \underline{результат}: $q \cdot w \cdot e$.
\end{itemize}


\problem Функция по заданной гипотенузе и катету, вычисляет другой катет прямоугольного треугольника.

\biglisting{../../Problems/src/problem-2-3.lisp}

\begin{itemize}
	\item $q \to q$;
	\item $w \to w$;
	\item[$\longrightarrow$] $(problem\!-\!2\!-\!3\; q\; w)$:
	\begin{itemize}
		\item[\textbullet] $leq \to q$;
		\item[\textbullet] $hypotenuse \to w$;
		\item[$\longrightarrow$] $\Bigl(\!sqrt\; \bigl(-\; (*\; hypotenuse\; hypotenuse)\; (*\; leg\; leg)\bigr)\!\Bigr)$:
		\begin{itemize}
			\item[$\longrightarrow$] $\bigl(-\; (*\; hypotenuse\; hypotenuse)\; (*\; leg\; leg)\bigr)$:
			\begin{itemize}
				\item[$\longrightarrow$] $(*\; hypotenuse\; hypotenuse)$:
				\begin{enumerate}
					\item[\textbullet] $hypotenuse \to hypotenuse$;
					\item[\textbullet] $hypotenuse \to hypotenuse$;
					\item[$\Longrightarrow$] применение <<$*$>> к $hypotenuse, hypotenuse$;
					\item[$\Longrightarrow$] \underline{результат}: $(hypotenuse)^2$.
				\end{enumerate}
				\item[$\longrightarrow$] $(*\; leg\; leg)$:
				\begin{enumerate}
					\item[\textbullet] $leg \to leg$;
					\item[\textbullet] $leg \to leg$;
					\item[$\Longrightarrow$] применение <<$*$>> к $leg, leg$;
					\item[$\Longrightarrow$] \underline{результат}: $(leg)^2$.
				\end{enumerate}
				\item[$\Longrightarrow$] применение <<$-$>> к $(hypotenuse)^2, (leg)^2$;
				\item[$\Longrightarrow$] \underline{результат}: $(hypotenuse)^2 - (leg)^2$.
			\end{itemize}
			\item[$\Longrightarrow$] применение $sqrt$ к $(hypotenuse)^2 - (leg)^2$;
			\item[$\Longrightarrow$] \underline{результат}: $\sqrt{(hypotenuse)^2 - (leg)^2}$.
		\end{itemize}
		\item[$\Longrightarrow$] применение $problem\!-\!2\!-\!3$ к $q, w$;
		\item[$\Longrightarrow$] \underline{результат}: $\sqrt{q^2 - w^2}$.
	\end{itemize}
\end{itemize}


\section{Вычислить результаты выражений}

\problem \hfill

\begin{lstlisting}
(list 'a 'b c) => Unbound variable C
\end{lstlisting}


\problem \hfill

\begin{lstlisting}
(cons 'a (b c)) => Unbound variable C
\end{lstlisting}


\problem \hfill

\begin{lstlisting}
(cons 'a '(b c)) => (A B C) 
\end{lstlisting}


\problem \hfill

\begin{lstlisting}
(caddr '(1 2 3 4 5)) => 3
\end{lstlisting}


\problem \hfill

\begin{lstlisting}
(cons 'a 'b 'c) => Too many arguments
\end{lstlisting}


\problem \hfill

\begin{lstlisting}
(list 'a (b c)) => Unbound variable C
\end{lstlisting}


\problem \hfill

\begin{lstlisting}
(list a '(b c)) => Unbound variable A
\end{lstlisting}


\problem \hfill

\begin{lstlisting}
(list (+ 1 (length '(1 2 3)))) => (4) 
\end{lstlisting}


\problem \hfill

\begin{lstlisting}
(cons 3 (list 5 6)) => (3 5 6)
\end{lstlisting}


\problem \hfill

\begin{lstlisting}
(list 3 'from 9 'gives (- 9 3)) => (3 FROM 9 GIVES 6) 
\end{lstlisting}


\problem \hfill

\begin{lstlisting}
(+ (length '(1 foo 2 too)) (car '(21 22 23))) => 25
\end{lstlisting}


\problem \hfill

\begin{lstlisting}
(cdr '(cons is short for ans)) => (IS SHORT FOR ANS)
\end{lstlisting}


\problem \hfill

\begin{lstlisting}
(car (list one two)) => Unbound variable ONE
\end{lstlisting}


\problem \hfill

\begin{lstlisting}
(cons 3 '(list 5 6)) => (3 LIST 5 6)
\end{lstlisting}


\problem \hfill

\begin{lstlisting}
(car (list 'one 'tow)) => ONE
\end{lstlisting}


\problem \hfill

\begin{lstlisting}
(list 'cons t NIL) => (CONST T NIL) 
\end{lstlisting}


\problem \hfill

\begin{lstlisting}
(eval (eval (list 'cons t NIL))) => Undefined function T
\end{lstlisting}


\problem \hfill

\begin{lstlisting}
(apply #'cons '(t NIL)) => (T) 
\end{lstlisting}


\problem \hfill
\begin{lstlisting}
(list 'eval NIL) => (EVAL NIL)
\end{lstlisting}


\problem \hfill
\begin{lstlisting}
(eval (list 'cons t nil)) => (T)
\end{lstlisting}


\problem \hfill
\begin{lstlisting}
(eval NIL) => NIL
\end{lstlisting}


\problem \hfill
\begin{lstlisting}
(eval (list 'eval NIL)) => NIL
\end{lstlisting}



\section{Написать функцию}

\problem Функция от двух списков-аргументов, которая возвращает \verb|T|, если первый аргумент имеет большую длину.

\biglisting{../../Problems/src/problem-4-1.lisp}


\problem Функция переводит температуру из системы Фаренгейта в температуру по Цельсию.

\biglisting{../../Problems/src/problem-4-2.lisp}



\section{Исследование функции}

Имеется следующая функция

\biglisting{../../Problems/src/problem-5.lisp}

\noindent
Необходимо вычислить результаты выражений.


\problem \hfill
\begin{lstlisting}
(mystery '(one two)) => (TWO ONE)
\end{lstlisting}


\problem \hfill
\begin{lstlisting}
(mystery 'free) => value FREE is not LIST
\end{lstlisting}


\problem \hfill
\begin{lstlisting}
(mystery (last 'one 'two)) => no list for LAST
\end{lstlisting}


\problem \hfill
\begin{lstlisting}
(mystery 'one 'two) => too many arguments
\end{lstlisting}