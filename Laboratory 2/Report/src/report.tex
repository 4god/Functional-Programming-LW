% !TEX root = main.tex

\section{Для указанных выражений составить диаграмму вычисления}

\vfill
\problem $\bigl(equal\; 3\; (abs -\!3)\bigr)$

\begin{itemize}
	\item[$\longrightarrow$]$\bigl(equal\; 3\; (abs -\!3)\bigr)$:
	\begin{itemize}
		\item[\textbullet] вычисляется $3 \to 3$;
		\item[$\longrightarrow$] $(abs -\!3)$:
		\begin{itemize}
			\item[\textbullet] вычисляется $-3 \to -3$;
			\item[$\Longrightarrow$] применение $abs$ к $-3$;
			\item[$\Longrightarrow$] \underline{возвращается результат}: $3$.
		\end{itemize}
		\item[$\Longrightarrow$] применение $equal$ к $3, 3$;
		\item[$\Longrightarrow$] \underline{возвращается результат}: T.
	\end{itemize}
\end{itemize}
\vfill


\problem $\bigl(equal\; (+\; 1\; 2)\; 3\bigr)$

\begin{itemize}
	\item[$\longrightarrow$] $\bigl(equal\; (+\; 1\; 2)\; 3\bigr)$:
	\begin{itemize}
		\item[$\longrightarrow$] $(+\; 1\; 2)$:
		\begin{itemize}
			\item[\textbullet] вычисляется $1 \to 1$;
			\item[\textbullet] вычисляется $2 \to 2$;
			\item[$\Longrightarrow$] применение <<$+$>> к $1, 2$;
			\item[$\Longrightarrow$] \underline{возвращается результат}: $3$.
		\end{itemize}
		\item[\textbullet] $3 \to 3$;
		\item[$\Longrightarrow$] применение $equal$ к $3, 3$;
		\item[$\Longrightarrow$] \underline{возвращается результат}: T.
	\end{itemize}
\end{itemize}
\vfill


\problem $\bigl(equal\; (*\; 4\; 7)\; 21\bigr)$

\begin{itemize}
	\item[$\longrightarrow$] $\bigl(equal\; (*\; 4\; 7)\; 21\bigr)$:
	\begin{itemize}
		\item[$\longrightarrow$] $(*\; 4\; 7)$:
		\begin{itemize}
			\item[\textbullet] вычисляется $4 \to 4$;
			\item[\textbullet] вычисляется $7 \to 7$;
			\item[$\Longrightarrow$] применение <<$*$>> к $4, 7$;
			\item[$\Longrightarrow$] \underline{возвращается результат}: $28$.
		\end{itemize}
		\item[\textbullet] вычисляется $21 \to 21$;
		\item[$\Longrightarrow$] применение $equal$ к $28, 21$;
		\item[$\Longrightarrow$] \underline{возвращается результат}: NIL.
	\end{itemize}
\end{itemize}


\newpage
\problem $\bigl(equal\; (*\; 2\; 3)\; (+\; 7\; 2) \bigr)$

\begin{itemize}
	\item[$\longrightarrow$] $\bigl(equal\; (*\; 2\; 3)\; (+\; 7\; 2) \bigr)$:
	\begin{itemize}
		\item[$\longrightarrow$] $(*\; 2\; 3)$:
		\begin{itemize}
			\item[\textbullet] вычисляется $2 \to 2$;
			\item[\textbullet] вычисляется $3 \to 3$;
			\item[$\Longrightarrow$] применение <<$*$>> к $2, 3$;
			\item[$\Longrightarrow$] \underline{возвращается результат}: $6$.
		\end{itemize}
		\item[$\longrightarrow$] $(+\; 7\; 2)$:
		\begin{itemize}
			\item[\textbullet] вычисляется $7 \to 7$;
			\item[\textbullet] вычисляется $2 \to 2$;
			\item[$\Longrightarrow$] применение <<$+$>> к $7, 2$;
			\item[$\Longrightarrow$] \underline{возвращается результат}: $9$.
		\end{itemize}
		\item[$\Longrightarrow$] применение $equal$ к $6, 9$;
		\item[$\Longrightarrow$] \underline{возвращается результат}: NIL.
	\end{itemize}
\end{itemize}


\problem $\bigl(equal\; (-\; 7\; 3)\; (*\; 3\; 2) \bigr)$

\begin{itemize}
	\item[$\longrightarrow$] $\bigl(equal\; (-\; 7\; 3)\; (*\; 3\; 2) \bigr)$:
	\begin{itemize}
		\item[$\longrightarrow$] $(-\; 7\; 3)$:
		\begin{itemize}
			\item[\textbullet] вычисляется $7 \to 7$;
			\item[\textbullet] вычисляется $3 \to 3$;
			\item[$\Longrightarrow$] применение <<$-$>> к $7, 3$;
			\item[$\Longrightarrow$] \underline{возвращается результат}: $4$.
		\end{itemize}
		\item[$\longrightarrow$] $(*\; 3\; 2)$:
		\begin{itemize}
			\item[\textbullet] вычисляется $3 \to 3$;
			\item[\textbullet] вычисляется $2 \to 2$;
			\item[$\Longrightarrow$] применение <<$*$>> к $3, 2$;
			\item[$\Longrightarrow$] \underline{вычисляется результат}: $6$.
		\end{itemize}
		\item[$\Longrightarrow$] применение $equal$ к $4, 6$;
		\item[$\Longrightarrow$] \underline{вычисляется результат}: NIL.
	\end{itemize}
\end{itemize}


\problem $\Bigl(equal\; \bigl(abs\; (-\; 2\; 4)\bigr)\; 3\Bigr)$

\begin{itemize}
	\item[$\longrightarrow$] $\Bigl(equal\; \bigl(abs\; (-\; 2\; 4)\bigr)\; 3\Bigr)$
	\begin{itemize}
		\item[$\longrightarrow$] $\bigl(abs\; (-\; 2\; 4)\bigr)$:
		\begin{itemize}
			\item[$\longrightarrow$] $(-\; 2\; 4)$:
			\begin{itemize}
				\item[\textbullet] вычисляется $2 \to 2$;
				\item[\textbullet] вычисляется $4 \to 4$;
				\item[$\Longrightarrow$] применение <<$-$>> к $2, 4$;
				\item[$\Longrightarrow$] \underline{вычисляется результат}: $-2$.
			\end{itemize}
			\item[$\Longrightarrow$] применение $abs$ к $-2$;
			\item[$\Longrightarrow$] \underline{вычисляется результат}: $2$.
		\end{itemize}
		\item[\textbullet] $3 \to 3$;
		\item[$\Longrightarrow$] применение $equal$ к $2, 3$;
		\item[$\Longrightarrow$] \underline{вычисляется результат}: NIL.
	\end{itemize}
\end{itemize}



\section{Написать функцию и составить для неё диаграмму вычисления}

\problem Функция вычисляет гипотенузу прямоугольного треугольника по заданным катетам.

\biglisting{../../Problems/src/problem-2-1.lisp}

\begin{itemize}
	\item[$\longrightarrow$] $(hypotenuse\!-\!rect\!-\!triangle\; q\; w)$
	\begin{itemize}
		\item[\textbullet] вычисляется $q \to q$;
		\item[\textbullet] вычисляется $w \to w$;
	\end{itemize}
	\item[$\Longrightarrow$] запускается функция $hypotenuse\!-\!rect\!-\!triangle$;
	\begin{itemize}
		\item[\textbullet] создаётся переменная $leg1$ cо значением $q$;
		\item[\textbullet] создаётся переменная $leg2$ cо значением $w$;
		\item[$\longrightarrow$] $\Bigl(sqrt\; \bigl(+\; (*\; leg1\; leg1)\; (*\; leg2\; leg2)\bigr)\Bigr)$:
		\begin{itemize}
			\item[$\longrightarrow$] $\bigl(+\; (*\; leg1\; leg1)\; (*\; leg2\; leg2)\bigr)$:
			\begin{enumerate}
				\item[$\longrightarrow$] $(*\; leg1\; leg1)$:
				\begin{enumerate}
					\item[\textbullet] вычисляется $leg1 \to q$;
					\item[\textbullet] вычисляется $leg1 \to q$;
				\end{enumerate}
				\item[$\Longrightarrow$] применение <<$*$>> к $q, q$;
				\item[$\Longrightarrow$] \underline{возвращается результат}: $q^2$.
				\item[$\longrightarrow$] $(*\; leg2\; leg2)$:
				\begin{enumerate}
					\item[\textbullet] $leg2 \to w$;
					\item[\textbullet] $leg2 \to w$;
				\end{enumerate}
				\item[$\Longrightarrow$] применение <<$*$>> к $w, w$;
				\item[$\Longrightarrow$] \underline{возвращается результат}: $w^2$.
			\end{enumerate}
			\item[$\Longrightarrow$] применение <<$+$>> к $q^2, w^2$;
			\item[$\Longrightarrow$] \underline{возвращается результат}: $q^2 + w^2$.
		\end{itemize}
		\item[$\Longrightarrow$] применение $sqrt$ к $q^2 + w^2$;
		\item[$\Longrightarrow$] \underline{возвращается результат}: $\sqrt{q^2 + w^2}$.
	\end{itemize}
	\item[$\Longrightarrow$] \underline{результат}: $\sqrt{q^2 + w^2}$.
\end{itemize}


\problem Функция вычисляет объём прямоугольного параллелепипеда по 3-м его сторонам.

\biglisting{../../Problems/src/problem-2-2.lisp}

\begin{itemize}
	\item[$\longrightarrow$] $(volume\!-\!rect\!-\!parallepiped\; q\; w\; e)$;
	\begin{itemize}
		\item[\textbullet] вычисляется $q \to q$;
		\item[\textbullet] вычисляется $w \to w$;
		\item[\textbullet] вычисляется $e \to e$;
		\item[$\Longrightarrow$] применение $volume\!-\!rect\!-\!parallepiped$ к $q$, $w$, $e$:
		\begin{itemize}
			\item[\textbullet] создаётся переменная $leg1$ со значением $q$;
			\item[\textbullet] создаётся переменная $leg2$ со значением $w$;
			\item[\textbullet] создаётся переменная $leg3$ со значением $e$;
			\item[$\longrightarrow$] $(*\; leg1\; leg2\; leg3)$:
			\begin{itemize}
				\item[\textbullet] вычисляется $leg1 \to q$;
				\item[\textbullet] вычисляется $leg2 \to w$;
				\item[\textbullet] вычисляется $leg3 \to e$;
				\item[$\Longrightarrow$] применение <<$*$>> к $q$, $w$, $e$;
				\item[$\Longrightarrow$] \underline{возвращается результат}: $q \cdot w \cdot e$.
			\end{itemize}
		\end{itemize}
		\item[$\Longrightarrow$] \underline{результат}: $q \cdot w \cdot e$.
	\end{itemize}
\end{itemize}


\problem Функция по заданной гипотенузе и катету, вычисляет другой катет прямоугольного треугольника.

\biglisting{../../Problems/src/problem-2-3.lisp}

\begin{itemize}
	\item $q \to q$;
	\item $w \to w$;
	\item[$\longrightarrow$] $(problem\!-\!2\!-\!3\; q\; w)$:
	\begin{itemize}
		\item[\textbullet] $leq \to q$;
		\item[\textbullet] $hypotenuse \to w$;
		\item[$\longrightarrow$] $\Bigl(\!sqrt\; \bigl(-\; (*\; hypotenuse\; hypotenuse)\; (*\; leg\; leg)\bigr)\!\Bigr)$:
		\begin{itemize}
			\item[$\longrightarrow$] $\bigl(-\; (*\; hypotenuse\; hypotenuse)\; (*\; leg\; leg)\bigr)$:
			\begin{itemize}
				\item[$\longrightarrow$] $(*\; hypotenuse\; hypotenuse)$:
				\begin{enumerate}
					\item[\textbullet] $hypotenuse \to hypotenuse$;
					\item[\textbullet] $hypotenuse \to hypotenuse$;
					\item[$\Longrightarrow$] применение <<$*$>> к $hypotenuse, hypotenuse$;
					\item[$\Longrightarrow$] \underline{результат}: $(hypotenuse)^2$.
				\end{enumerate}
				\item[$\longrightarrow$] $(*\; leg\; leg)$:
				\begin{enumerate}
					\item[\textbullet] $leg \to leg$;
					\item[\textbullet] $leg \to leg$;
					\item[$\Longrightarrow$] применение <<$*$>> к $leg, leg$;
					\item[$\Longrightarrow$] \underline{результат}: $(leg)^2$.
				\end{enumerate}
				\item[$\Longrightarrow$] применение <<$-$>> к $(hypotenuse)^2, (leg)^2$;
				\item[$\Longrightarrow$] \underline{результат}: $(hypotenuse)^2 - (leg)^2$.
			\end{itemize}
			\item[$\Longrightarrow$] применение $sqrt$ к $(hypotenuse)^2 - (leg)^2$;
			\item[$\Longrightarrow$] \underline{результат}: $\sqrt{(hypotenuse)^2 - (leg)^2}$.
		\end{itemize}
		\item[$\Longrightarrow$] применение $problem\!-\!2\!-\!3$ к $q, w$;
		\item[$\Longrightarrow$] \underline{результат}: $\sqrt{q^2 - w^2}$.
	\end{itemize}
\end{itemize}


\problem Функция вычисляет площадь трапеции по её основаниям и высоте

\biglisting{../../Problems/src/problem-2-4.lisp}

\begin{itemize}
	\item $a \to a$;
	\item $b \to b$;
	\item $h \to h$;
	\item[$\longrightarrow$] $(trapezoid\!-\!area\; a\; b\; h)$:
	\begin{itemize}
		\item[\textbullet] $parallel\!-\!side1 \to a$;
		\item[\textbullet] $parallel\!-\!side2 \to b$;
		\item[\textbullet] $height \to h$;
		\item[$\longrightarrow$] $\bigl(*\; 0.5\; height\; (+\; parallel\!-\!side1\; parallel\!-\!side2)\bigr)$:
		\begin{itemize}
			\item[\textbullet] $0.5 \to 0.5$;
			\item[\textbullet] $height \to height$;
			\item[$\longrightarrow$] $(+\; parallel\!-\!side1\; parallel\!-\!side2)$:
			\begin{itemize}
				\item[\textbullet] $parallel\!-\!side1 \to parallel\!-\!side1$;
				\item[\textbullet] $parallel\!-\!side2 \to parallel\!-\!side2$;
				\item[$\Longrightarrow$] применение <<$+$>> к $parallel\!-\!side1, parallel\!-\!side2$;
				\item[$\Longrightarrow$] \underline{результат}: $parallel\!-\!side1 + parallel\!-\!side2$.
			\end{itemize}
			\item[$\Longrightarrow$] применение <<$*$>> к $0.5,\; height,\; parallel\!-\!side1 + parallel\!-\!side2$;
			\item[$\Longrightarrow$] \underline{результат}: $0.5 \cdot height \cdot (parallel\!-\!side1 + parallel\!-\!side2)$.
		\end{itemize}
		\item[$\Longrightarrow$] применение $trapezoid\!-\!area$ к $a,\; b,\; h$;
		\item[$\Longrightarrow$] \underline{результат}: $0.5 \cdot h \cdot (a + b)$.
	\end{itemize}
\end{itemize}



\section{Вычислить результаты выражений}

\problem \hfill

\begin{lstlisting}
(list 'a 'b c) => Unbound variable C
\end{lstlisting}


\problem \hfill

\begin{lstlisting}
(cons 'a (b c)) => Unbound variable C
\end{lstlisting}


\problem \hfill

\begin{lstlisting}
(cons 'a '(b c)) => (A B C) 
\end{lstlisting}


\problem \hfill

\begin{lstlisting}
(caddr '(1 2 3 4 5)) => 3
\end{lstlisting}


\problem \hfill

\begin{lstlisting}
(cons 'a 'b 'c) => Too many arguments
\end{lstlisting}


\problem \hfill

\begin{lstlisting}
(list 'a (b c)) => Unbound variable C
\end{lstlisting}


\problem \hfill

\begin{lstlisting}
(list a '(b c)) => Unbound variable A
\end{lstlisting}


\problem \hfill

\begin{lstlisting}
(list (+ 1 (length '(1 2 3)))) => (4) 
\end{lstlisting}


\problem \hfill

\begin{lstlisting}
(cons 3 (list 5 6)) => (3 5 6)
\end{lstlisting}


\problem \hfill

\begin{lstlisting}
(list 3 'from 9 'gives (- 9 3)) => (3 FROM 9 GIVES 6) 
\end{lstlisting}


\problem \hfill

\begin{lstlisting}
(+ (length '(1 foo 2 too)) (car '(21 22 23))) => 25
\end{lstlisting}


\problem \hfill

\begin{lstlisting}
(cdr '(cons is short for ans)) => (IS SHORT FOR ANS)
\end{lstlisting}


\problem \hfill

\begin{lstlisting}
(car (list one two)) => Unbound variable ONE
\end{lstlisting}


\problem \hfill

\begin{lstlisting}
(cons 3 '(list 5 6)) => (3 LIST 5 6)
\end{lstlisting}


\problem \hfill

\begin{lstlisting}
(car (list 'one 'tow)) => ONE
\end{lstlisting}


\problem \hfill

\begin{lstlisting}
(list 'cons t NIL) => (CONST T NIL) 
\end{lstlisting}


\problem \hfill

\begin{lstlisting}
(eval (eval (list 'cons t NIL))) => Undefined function T
\end{lstlisting}


\problem \hfill

\begin{lstlisting}
(apply #'cons '(t NIL)) => (T) 
\end{lstlisting}


\problem \hfill
\begin{lstlisting}
(list 'eval NIL) => (EVAL NIL)
\end{lstlisting}


\problem \hfill
\begin{lstlisting}
(eval (list 'cons t nil)) => (T)
\end{lstlisting}


\problem \hfill
\begin{lstlisting}
(eval NIL) => NIL
\end{lstlisting}


\problem \hfill
\begin{lstlisting}
(eval (list 'eval NIL)) => NIL
\end{lstlisting}



\section{Написать функцию}

\problem Функция от двух списков-аргументов, которая возвращает \verb|T|, если первый аргумент имеет большую длину.

\biglisting{../../Problems/src/problem-4-1.lisp}


\problem Функция переводит температуру из системы Фаренгейта в температуру по Цельсию.

\biglisting{../../Problems/src/problem-4-2.lisp}



\section{Исследование функции}

Имеется следующая функция

\biglisting{../../Problems/src/problem-5.lisp}

\noindent
Необходимо вычислить результаты выражений.


\problem \hfill
\begin{lstlisting}
(mystery '(one two)) => (TWO ONE)
\end{lstlisting}


\problem \hfill
\begin{lstlisting}
(mystery 'free) => value FREE is not LIST
\end{lstlisting}


\problem \hfill
\begin{lstlisting}
(mystery (last 'one 'two)) => no list for LAST
\end{lstlisting}


\problem \hfill
\begin{lstlisting}
(mystery 'one 'two) => too many arguments
\end{lstlisting}



\section{Заключение}

В данной лабораторной работе было рассмотрено составление \textit{диаграммы вычисления}, а так же изучены функции \verb|EQUAL|, \verb|ABS|, \verb|SQRT|, \verb|LENGTH|, \verb|EVAL|, \verb|APPLY|, \verb|FIRST|, \verb|SECOND|.